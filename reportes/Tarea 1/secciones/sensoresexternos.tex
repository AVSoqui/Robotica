\section{Sensores Externos}
Los sensores externos se utilizan principalmente para saber más acerca del ambiente del robot, especialmente sobre los objetos que se va a manipular. Los sensores externos pueden
dividirse en las siguientes categorías:

\subsection{Tipo de Contacto}

\addcontentsline{toc}{subsubsection}{Interruptor de límite}
\subsection*{\quad\textbf{Interruptor de límite}}
El interruptor de límite tiene generalmente un brazo mecánico sensible a la presión. Cuando un objeto aplica presión sobre el brazo mecánico, se activa el interruptor. Es posible que un objeto tenga un imán que cause que un contacto suba y cierre cuando el objeto pase sobre el brazo. El registro de subida mantiene la señal en +V hasta que el interruptor cierra, transmitiendo la señal a tierra.\linebreak
Los interruptores de límite pueden ser normalmente abiertos (NO) o normalmente cerrados (NC) y pueden tener polos múltiples. Un interruptor normalmente abierto permite flujo de corriente hasta que se aplica presión al interruptor. 

\addcontentsline{toc}{subsubsection}{Sensor de Proximidad}
\subsection*{\quad\textbf{Sensor de Proximidad}}
La detección de proximidad es la técnica que se usa para detectar la presencia o ausencia de un objeto por medio de un sensor electrónico sin contacto. Hay dos tipos de sensores de proximidad: inductivo y capacitivo. Los sensores de proximidad inductivos se usan en lugar de interruptores de límite para la detección sin contacto de objetos metálicos. Los sensores de proximidad capacitivos se usan sobre la misma base que los sensores de proximidad inductivos, pero también pueden detectar objetos no metálicos.




\cite{saha2014robotica}
\pagebreak
