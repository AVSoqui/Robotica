\section{\textbf{Introducción}}

El avance de la robótica ha sido posible en gran medida gracias a la integración de sensores que permiten a los robots percibir y responder a su entorno de manera eficiente. Los sensores son dispositivos esenciales que recopilan información sobre diversas variables y la convierten en señales procesables por el sistema de control del robot. Gracias a esta retroalimentación, los robots pueden mejorar su precisión, autonomía y capacidad de adaptación en diferentes entornos y tareas.

Los sensores utilizados en robótica pueden clasificarse en dos grandes categorías: sensores internos y sensores externos.

\subsection*{\quad\textbf{Sensores internos}}
Son aquellos que miden parámetros relacionados con el estado interno del robot, como la posición y velocidad de sus articulaciones, la aceleración, la fuerza aplicada por sus motores y el nivel de carga de su batería. Estos sensores son cruciales para garantizar el correcto funcionamiento del sistema mecánico y electrónico, evitando errores, optimizando el rendimiento y facilitando el mantenimiento preventivo. Algunos ejemplos incluyen encoders, giroscopios y sensores de corriente.

\subsection*{\quad\textbf{Sensores externos}}
Se encargan de captar información del entorno en el que opera el robot, permitiéndole interactuar con los objetos y adaptarse a diferentes condiciones. Estos sensores incluyen cámaras, sensores de proximidad, LIDAR, ultrasonidos y micrófonos, entre otros. Son fundamentales en aplicaciones como la navegación autónoma, la detección de obstáculos, el reconocimiento de patrones y la interacción con humanos.

\vspace{0.5cm}

El uso combinado de sensores internos y externos permite a los robots realizar tareas con un alto grado de precisión y eficiencia, ya sea en la automatización industrial, la robótica de servicio, la exploración espacial, la medicina o la robótica móvil. En este reporte, se analizarán en detalle los diferentes tipos de sensores empleados en robótica, su funcionamiento, sus aplicaciones y los desafíos asociados con su integración en sistemas autónomos.

A través de este estudio, se busca comprender la importancia de los sensores en la evolución de la robótica y cómo su desarrollo continúa impulsando la creación de robots cada vez más inteligentes, adaptables y funcionales en distintos campos de la tecnología.
\pagebreak